\subsection*{Семинар 2}
\task{1}

$$45x - 37y = 25$$

$$45x - 37y = 1$$

$$
\begin{array}{c}
    45 = 1 \cdot 37 + \colorbox{red}{8}\\
    \colorbox{yellow}{37} = \colorbox{yellow}{$4 \cdot 8$} + \colorbox{red}{5}\\
    \colorbox{yellow}{8} = \colorbox{yellow}{$1 \cdot 5$} + \colorbox{red}{3}\\
    \colorbox{yellow}{5} = \colorbox{yellow}{$1 \cdot 3$} + \colorbox{red}{2}\\
    \colorbox{yellow}{3} = \colorbox{yellow}{$1 \cdot 2$} + \colorbox{red}{1}\\
    2 = 1 \cdot 1 + 0\\
\end{array}$$

Теперь просто выражаем красные через синие и подставляем в уравнение, так доходим до самого верха.

$$\colorbox{red}{1} = \colorbox{yellow}{$1 \cdot 3$} - \colorbox{yellow}{$1 \cdot 2$} = 1 \cdot 3 - (5 - 3) = -1 \cdot 5 + 2 \cdot 3 = -1 \cdot 5 + 2 \cdot (8 - 5) = -3 \cdot 5 + 2 \cdot 8 =$$

$$= -3 \cdot (37 - 4 \cdot 8) + 2 \cdot 8 = - 3 \cdot 37 + 14 \cdot 8 = -17 \cdot 37 + 14 \cdot 45$$

$$\hat{x_0} = 14, \hat{y_0} = 17$$

$$x_0 = 14 \cdot 25, y_0 = 17 \cdot 25$$

$$\begin{cases}
    x = x_0 + 37t\\
    y = y_0 + 45t\\
\end{cases}$$

$$\begin{cases}
    x = 14 \cdot 25 + 37t\\
    y = 17 \cdot 25 + 45t\\
\end{cases}$$

\task{2}

\subtask{а}

$$\sqrt{3} = 1 + \sqrt{3} - 1 = 1 + \frac{1}{\frac{1}{ \sqrt{3} - 1}} = 1 + \frac{1}{\frac{ \sqrt{3} + 1}{2}} = 1 + \frac{1}{1 + \frac{ \sqrt{3} - 1}{2}} =$$

$$= 1 + \frac{1}{1 + \frac{1}{\frac{2}{ \sqrt{3} - 1}}} = 1 + \frac{1}{1 + \frac{1}{ \sqrt{3} + 1}} = 1 + \frac{1}{1 + \frac{1}{ 2 + \sqrt{3} - 1}} = [1, \overline{1, 2}]$$

\subtask{б}

$$\frac{1}{2} + \sqrt{7} = 3 + \sqrt{7} - \frac{5}{2} = 3 + \frac{1}{\frac{1}{\sqrt{7} - \frac{5}{2}}} = 3 + \frac{1}{\frac{4\sqrt{7} + 10}{3}} = $$

$$= 3 + \frac{1}{6 + \frac{4\sqrt{7} - 8}{3}} = 3 + \frac{1}{6 + \frac{1}{\frac{3}{4\sqrt{7} - 8}}} = 3 + \frac{1}{6 + \frac{1}{\frac{\sqrt{7} + 2}{4}}} = 3 + \frac{1}{6 + \frac{1}{1 + \frac{\sqrt{7} - 2}{4}}} =$$

$$= 3 + \frac{1}{6 + \frac{1}{1 + \frac{1}{\frac{4(\sqrt{7} + 2)}{3}}}}= 3 + \frac{1}{6 + \frac{1}{1 + \frac{1}{6 + \frac{4\sqrt{7} - 10}{3}}}} = 3 + \frac{1}{6 + \frac{1}{1 + \frac{1}{6 + \frac{1}{\frac{3}{4\sqrt{7} - 10}}}}} = 3 + \frac{1}{6 + \frac{1}{1 + \frac{1}{6 + \frac{1}{\frac{3(4\sqrt{7} + 10)}{12}}}}} =$$

$$= 3 + \frac{1}{6 + \frac{1}{1 + \frac{1}{6 + \frac{1}{\frac{2\sqrt{7} + 5}{2}}}}} = 3 + \frac{1}{6 + \frac{1}{1 + \frac{1}{6 + \frac{1}{5 + \frac{2\sqrt{7} - 5}{2}}}}} = [3, \overline{6,1,6,5}]$$

\task{3}

\subtask{а}

$$x = 1 + \frac{1}{2 + \frac{1}{3 + \frac{1}{1 + ...}}}$$

$$x = 1 + \frac{1}{2 + \frac{1}{3 + \frac{1}{x}}} = 1 + \frac{1}{2 + \frac{x}{3x + 1}} = 1 + \frac{3x + 1}{6x + 2 + x} = 1 + \frac{3x + 1}{7x + 2} = \frac{3x + 1 + 7x + 2}{7x + 2} = \frac{10x + 3}{7x + 2}$$

$$x (7x + 2) = 10x + 3$$
$$7x^2 + 2x = 10x + 3$$
$$7x^2 - 8x - 3 = 0$$
$$d = 64 + 84 = 148$$

$$\sqrt{d} = 2\sqrt{37}$$

$$x = \frac{8 +- 2\sqrt{37}}{14} \Rightarrow x = \frac{4 + \sqrt{37}}{7}$$

\subtask{б}

Аналогично с пунктом а просто посчитать дробь $[5, \overline{1, 4}]$

\task{4}

\task{5}

$$\sqrt{2} < \frac{m}{n} < \frac{297}{210}$$

$$\sqrt{2} - 1 < \frac{m}{n} - 1 < \frac{87}{210}$$
$$\frac{210}{87} < \frac{1}{\frac{m}{n} - 1} < \frac{1}{\sqrt{2} - 1} = \sqrt{2} + 1$$

\task{6}

\task{7}

\task{8}

\task{9}

\task{10}
