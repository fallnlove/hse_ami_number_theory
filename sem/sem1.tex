\subsection*{Семинар 1}

\task{1}

\task{2}

$$(5a + 3b, 13a + 8b) = (5a + 3b, 3a + 2b) = (2a + b, a + b) = (a, a + b) = (a, b)$$

\task{3}

$$(\underbrace{1...1}_{m}, \underbrace{1...1}_{n}) = (\underbrace{1...1}_{m - n} \underbrace{0...0}_{n}, \underbrace{1...1}_{n}) =$$

$$= (\underbrace{1...1}_{m - n}, \underbrace{1...1}_{n}) = \underbrace{1...1}_{(n, m)}$$

\task{4}

Пункт а

$$\frac{21n + 4}{14n + 3}$$

$$(21n + 4, 14n + 3) = (7n + 1, 1) = 1$$

Пункт б

$$\frac{n^2 - n + 1}{n^2 + 1}$$

$$(n^2 - n + 1, n^2 + 1) = (-n, n^2 + 1) = (n, n^2 + 1) = 1$$

\task{5}

$$n \geq m$$

$$(a^n - 1, a^m - 1) = (a^n - 1 - (a^m - 1)a^{n - m}, a^m - 1) = (a^{n - m} - 1, a^m - 1)$$

Далее повторяем такую операцию пока не придем к $(a^{(n, m)} - 1, a^{(n, m)} - 1)$.

Еще можно сослаться на задачу 3, просто в $a$-ичной системе счисления

\task{6}

Зафиксируем простое число $p$, по основное теореме арифметики, нод представляется в виде произведения простых, тогда очевидно будут выполнены
наши равенства а и б. Чтобы доказать пункт с можно просто втупую перемножить и получить тоже самое, просто в представлении простых.

\task{7}

Пункт а

$[a, (a, b)] = a$, так как по определению $(a, b)$ он делит $a$, а меньше взять нельзя.

Пункт б

$abc = [a, b, c](ab, ac, bc)$

Зафиксируем простое $p$,

$$[a, b, c] = p^{max(\alpha_1, \alpha_2, \alpha_3)}$$

$$(ab, ac, bc) = p^{min(\alpha_1 + \alpha_2, \alpha_1 + \alpha_3, \alpha_2 + \alpha_3)}$$

Тогда

$$[a, b, c](ab, ac, bc) = p^{max(\alpha_1, \alpha_2, \alpha_3)} \cdot p^{min(\alpha_1 + \alpha_2, \alpha_1 + \alpha_3, \alpha_2 + \alpha_3)} = p^{\alpha_1 + \alpha_2 + \alpha_3} = abc$$

\task{8}

Заметим, что количество чисел, которые делит $p$ на отрезке $[1, n]$ равно в точности $\lfloor \frac{n}{p} \rfloor$, доказать можно индукцией.

Для понимания просто заметим, что делятся на $p$ числа вида $k \cdot p$, $p, 2p, 3p, ...$

Чтобы найти степень $p$, нужно просто проссумировать по всем степеням $p$ наши округленные значения. Почему так? Ну потому что для 1 степени, мы посчитаем
все вхождения первой степени, для 2 степени мы посчитаем количество чисел куда оно входит, поэтому посчитаем вхождения 2 степени, и все будет нормально
так как 1 степень уже учли. Теперь это можно пруфануть индукцией.

\textbf{Обозначение} $p^a || m$ - $p^a$ делит $m$, $a$ - такое максимально

\task{9}

Воспользуемся 8 задачей. Найдем какое количество раз $p$ входит в разложение $n!, (n - m)!, m!$.

Переведем в $p$-ичную систему исчисления, разрежем по $k$ разрядам.

\begin{equation*}
    \begin{cases}
        p^{\Sum{}{}z_k} || n!\\
        p^{\Sum{}{}x_k} || m!\\
        p^{\Sum{}{}y_k} || (n - m)!\\
    \end{cases}
\end{equation*}

Тут $x_k$ - k префикс числа m в $p$-ичной системе счисления, $x_k =$ $k$ слагаемому в разложении $\alpha_p$, так как мы в $p$-ичной системе счисления.

$$m = x...a_1a_0$$

$$n - m = y...b_1b_0$$

$$n = z...c_1c_0$$

$p^{\Sum{k}{}(z_k - (x_k + y_k))}$ - максимальное вхождение в $\binom{n}{m}$

Посмотрим как мы складывали в столбик, так как $x_k$ - префикс, то разность в $z_k - (x_k + y_k) = 1$, только если был перенос разряда.

\task{10}

$$d_n = \left(\binom{n}{1}, \binom{n}{2}, ..., \binom{n}{n - 1}\right)$$

$\binom{n}{1} = n$, n - делится на $d_n$

Пусть есть какой-то ответ $p^s||n$

Рассмотрим $\binom{n}{p^s}$, если $n \neq p^s$.

$$p^s = 1\underbrace{0...0}^{s}$$

$$n - p^s = ... a ... p - 1\underbrace{0...0}^{s}$$
$$n = 1\underbrace{0...0}^{s + 1}$$

Тогда $p$ не делит $\binom{n}{p^s}$. Значит ответ 1

Если $n = p^s$

$$p^{s - 1} = 1\underbrace{0...0}^{s - 1}$$

$$n - p^{s - 1} = p - 1\underbrace{0...0}^{s - 1}$$
$$n = 1\underbrace{0...0}^{s}$$

Если $n = p^{\alpha}$, $p$ - простое, то ответ $p$, иначе 1.
