\subsection{Семинар 3}

\task{1}

\subtask{а)}

\[\alpha = [1, 2, 4, 8, 16, 32, 64, 128]\]

\[\begin{array}{|c|c|c|c|c|c|c|}
    \hline
    a & 1 & 2 & 4 & 8  & 16 & 32\\
    \hline
    p & 1 & 3 & 13& 107& 1725&55307\\
    \hline
    q & 1 & 2 & 9 & 74 & 1193& 38250\\
    \hline
\end{array}\]

Возьмем до 4 элемента включая, так как нам хватит точности $\frac{1}{1193 \cdot 74}$ итого приближение равно $\frac{107}{74}$.

\subtask{б)}

\[\alpha = \sqrt{5} + 2\]

\[2+ \sqrt{5} = 4 + {1 \ove {1 \ove \sqrt{5} - 2}} = 4 + {1 \ove {\sqrt{5} + 2 \ove 1}} = [4; \overline{4}]\]

\[\begin{array}{|c|c|c|c|c|c|c|c|}
    \hline
    a &  &  4 & 4 & 4 & 4 & 4 & 4\\
    \hline
    p & 1 & 4 & 17& 72& 305&1292 & ...\\
    \hline
    q & 0 & 1 & 4 & 17 & 72& 305 & 1292\\
    \hline
\end{array}\]

Так как $72 \cdot 305 > 10000$, то нужное приближение равно $\frac{305}{72}$.

Оценим разность оценки и самого числа сверху и снизу

\[\frac{1}{72 \cdot 305} \geq |\alpha - \frac{305}{72}| \geq \frac{a_{k + 2}}{q_k q_{k + 2}} = \frac{4}{72 \cdot 1292}\]

\task{2}

\[\sqrt{n} = [a_0; \overline{a_1, a_2, ..., a_2, a_1, 2a_0}]\]

\[a_0 + \sqrt{n} = [\overline{2a_0, a_1, a_2, ..., a_2, a_1}]\]

\[\frac{x}{y} = \frac{p_{k - 1}}{q_{k - 1}} = [a_0; a_1, a_2, ..., a_2, a_1]\]

\[x^2 -dy^2 = \pm 1\]

\[(x - \sqrt{d}y)(x + \sqrt{d}y) = \pm 1\]

\[(\frac{x}{y} - \sqrt{d})(\frac{x}{y} + \sqrt{d}) = \pm \frac{1}{y^2}\]

\[|\frac{x}{y} - \sqrt{d}| \leq \frac{1}{y^2(\sqrt{d} + 1)} \leq \frac{1}{2y^2}\]

По теореме с лекции понимаем, что решениями будут числа являющиеся подходящими к $\sqrt{d}$ и достаточная точность будет, если оборвать
дробь при $2a_0$, это мы пока принимаем без доказательства.

\subtask{а)}

\[\sqrt{10} = 3 + {1 \ove {\sqrt{10} + 3 \ove 1}} = [3; \overline{6}] \Rightarrow k = 1 \Rightarrow \delta = 2\]

\[3^2 - 10 \cdot 1^2 = -1\]

\[\begin{array}{|c|c|c|c|}
    \hline
    a &  &  3 & 6\\
    \hline
    p & 1 & 3 & 19\\
    \hline
    q & 0 & 1 & 6\\
    \hline
\end{array}\]

\[(x_1, y_1) = (p_1, q_1) = (19, 6)\]
\[\dots\]
\[(x_n, y_n) = (p_{2n - 1}, q_{2n - 1})\]

\[x_n + \sqrt{d}y_n = (p_{k - 1} + q_{k - 1}\sqrt{d})^{\delta n} = (3 + 1 \cdot \sqrt{10})^{2n} = (19 + 6 \sqrt{10})^n\]

\[x_n + \sqrt{d}y_n = (19 + 6 \sqrt{10})^n\]

\[x_n - \sqrt{d}y_n = (19 - 6 \sqrt{10})^n\]

\[x_n = \frac{1}{2}((19 + 6 \sqrt{10})^n + (19 - 6 \sqrt{10})^n)\]

Почему все решения в натуральных чисел представимы именно так и мы не потеряли других? Поверим теореме из условия, пока не доказываем.

\subtask{б)}

\[\sqrt{13} = 3 + {1 \ove {\sqrt{13} + 3 \ove 4}} = 3 + {1 \ove 1 + { 1 \ove {\sqrt{13} + 1 \ove 3}}} = 3 + {1 \ove 1 + { 1 \ove 1 + {1 \ove {\sqrt{13} + 2 \ove 3}}}} =\]

\[= 3 + {1 \ove 1 + { 1 \ove 1 + {1 \ove 1 + {\sqrt{13} - 1 \ove 3}}}} = 3 + {1 \ove 1 + { 1 \ove 1 + {1 \ove 1 + {1 \ove {\sqrt{13} + 1 \ove 4}}}}}= [3; \overline{1, 1, 1, 1, 6}] \Rightarrow k = 5 \Rightarrow \delta = 2\]

Тут мне стало лень техать, это глина, можете сами проверить руками.

\[3^2 - 10 \cdot 1^2 = -1\]

\[\begin{array}{|c|c|c|c|c|c|c|c|}
    \hline
    a &  &  3 & 1 & 1 & 1 & 1 & 6\\
    \hline
    p & 1 & 3 & 4 & 7 & 11 & 18 & 119\\
    \hline
    q & 0 & 1 & 1 & 2 & 3 & 5 & 33\\
    \hline
\end{array}\]

\[x_n + \sqrt{13}y_n = (18 + 5 \sqrt{13})^{2n}\]

\task{3}

Так как мы рассматриваем $c > 2$ воспользуемся Теоремой Лежандра.

\[\alpha = [a_0; a_1, ..., a_n, \alpha_{n + 1}] \ \ \ \ \ \ \sqrt{2} = [1; \overline{2}]\]

\[\alpha_{n + 1} = \sqrt{2} + 1\]

\[|\alpha - \frac{p_n}{q_n}| = \frac{1}{q_n(\alpha_{n + 1}q_n + q_{n - 1})} = \frac{1}{q^2(\sqrt{2} + 1 + \frac{q_{n - 1}}{q_n})}\]

\[\frac{q_{n - 1}}{q_n} = \frac{q_{n - 1}}{2q_{n - 1} + q_{n - 2}} = \frac{1}{2 + \frac{q_{n - 2}}{q_{n - 1}}}\]


По индукции докажем, что для номеров одной четности $\frac{q_{n - 2}}{q_{n - 1}} \leq \sqrt{2} - 1$, пруфаните сами или поверьте Кириллу Кудряшову.
Для другой четности неравенство обратное.

\[\frac{1}{q^2(\sqrt{2} + 1 + \frac{q_{n - 1}}{q_n})} \geq \frac{1}{q^2(\sqrt{2} + 1 + \sqrt{2} - 1)} = \frac{1}{2\sqrt{2}q^2}\]

Доказали первый пункт.

\[\lim\limits_{n \to \infty} \frac{q_{n - 1}}{q_n} = \sqrt{2} - 1\]

\[|\frac{q_{n - 1}}{q_n} - (\sqrt{2} - 1)| \leq \frac{1}{q_n^2}\]

Поэтому если мы возьмем $c > 2 \sqrt{2}$, то начиная с некоторого числа решений не будет, так как у нас есть предел.
